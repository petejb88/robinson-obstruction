\documentclass[11pt,reqno
,draft
% ,final
]{amsart}

\pdfcompresslevel=0
\pdfobjcompresslevel=0

\usepackage{xr-hyper}
\usepackage[pagebackref, colorlinks, citecolor=PineGreen, linkcolor=PineGreen]{hyperref}
\hypersetup{
  final,
  pdftitle={Equivariant Robinson obstruction theory},
  pdfauthor={Belmont, Eva and Bonventre, P. and Li, Ang},
  linktoc=page
}



% ---- Commands on draft --------

\usepackage[dvipsnames]{xcolor}% adds colors
\usepackage{ifdraft}
\ifdraft{
  %\color[RGB]{63,63,63}
  %\pagecolor[RGB]{220,220,204}
  \usepackage[notref,notcite]{showkeys}
  \usepackage{todonotes}
}
{
  \usepackage[disable]{todonotes}
}

\usepackage{amsmath, amsthm}% {amsfonts, amssymb}


% ------ New Characters --------------------------------------

\usepackage[latin1]{inputenc}%
\usepackage[T1]{fontenc}
\usepackage{MnSymbol}
\usepackage[
cal = cm,
bb = ams,
frak = euler,
scr = rsfs
]{mathalpha}

\usepackage[normalem]{ulem}% underlining
\usepackage{bbm}% more bb

%\usepackage{dsfont}% double strike-through
% \usepackage{upgreek}



%----- Enumerate ---------------------------------------------
\usepackage[inline,shortlabels]{enumitem}% % can use \begin{enumerate*} for inparaenum


% ---------- Page Typesetting ----------
\usepackage[final]{microtype}
\usepackage{relsize}
\usepackage[margin=1in]{geometry}


%-------- Tikz ---------------------------
\usepackage{tikz}%
\usetikzlibrary{matrix,arrows,decorations.pathmorphing,cd,patterns,calc,backgrounds,patterns}
\tikzset{%
  dummy/.style    = {circle,draw,inner sep=0pt,minimum size=2mm},%
}



% ----- Labels Changed? --------

\makeatletter

\def\@testdef #1#2#3{%
  \def\reserved@a{#3}\expandafter \ifx \csname #1@#2\endcsname
  \reserved@a  \else
  \typeout{^^Jlabel #2 changed:^^J%
    \meaning\reserved@a^^J%
    \expandafter\meaning\csname #1@#2\endcsname^^J}%
  \@tempswatrue \fi}

\makeatother


%%%%%%%%%%%%%%%%%%%%%%%%% INTERNAL REFERENCES %%%%%%%%%%%%%%%%%%%%%%%%%%%%%%%%%%%

\numberwithin{equation}{section} 
\numberwithin{figure}{section}


% ------- New Theorems/ Definition/ Names-----------------------

\usepackage[nameinlink,capitalise,noabbrev]{cleveref}

%%%% to get \cref to behave as \eqref
\crefname{equation}{}{}

% \theoremstyle{plain} % bold name, italic text
\newtheorem{theorem}[equation]{Theorem}%
\newtheorem*{theorem*}{Theorem}%
\newtheorem{lemma}[equation]{Lemma}%
\newtheorem{proposition}[equation]{Proposition}%
\newtheorem{corollary}[equation]{Corollary}%
\newtheorem{conjecture}[equation]{Conjecture}%
\newtheorem*{conjecture*}{Conjecture}%
\newtheorem{claim}[equation]{Claim}%
\newtheorem{question}[equation]{Question}


\theoremstyle{definition} % bold name, plain text
\newtheorem{definition}[equation]{Definition}%
\newtheorem*{definition*}{Definition}%
\newtheorem{example}[equation]{Example}%
\newtheorem{remark}[equation]{Remark}%
\newtheorem{notation}[equation]{Notation}%
\newtheorem{convention}[equation]{Convention}%
\newtheorem{assumption}[equation]{Assumption}%


% %%%%%%%%%%%%%%%%%%%%%%%%%%%%%%%%%%%%%%%%%%%%%%%%%%%%%%%%%%%%%%%%%%%%%%%%%%%%%%%%
% ------------------------------ COMMANDS ------------------------------

% ---------- macros

\newcommand{\set}[1]{\left\{#1\right\}}%
\newcommand{\sets}[2]{\left\{ #1 \;|\; #2\right\}}%
\newcommand{\longto}{\longrightarrow}%
\newcommand{\into}{\hookrightarrow}%
\newcommand{\onto}{\twoheadrightarrow}%

\usepackage{harpoon}
\newcommand{\vect}[1]{\text{\overrightharp{\ensuremath{#1}}}}


% ---------- operators

\newcommand{\Sym}{\ensuremath{\mathsf{Sym}}}%
\newcommand{\Fin}{\mathsf{F}}%
\newcommand{\Set}{\ensuremath{\mathsf{Set}}}
\newcommand{\Top}{\ensuremath{\mathsf{Top}}}
\newcommand{\sSet}{\ensuremath{\mathsf{sSet}}}%
\newcommand{\Cat}{\mathsf{Cat}}
\newcommand{\sCat}{\mathsf{sCat}}
\newcommand{\Op}{\mathsf{Op}}%
\newcommand{\sOp}{\ensuremath{\mathsf{sOp}}}%
\newcommand{\fgt}{\ensuremath{\mathsf{fgt}}}%
\newcommand{\dSet}{\mathsf{dSet}}
\newcommand{\Fun}{\mathsf{Fun}}
\newcommand{\Fib}{\mathsf{Fib}}
\newcommand{\Alg}{\mathsf{Alg}}
\newcommand{\Kl}{\mathsf{Kl}}



\DeclareMathOperator{\hocmp}{hocmp}%
\DeclareMathOperator{\cmp}{cmp}%
\DeclareMathOperator{\hofiber}{hofiber}%
\DeclareMathOperator{\fiber}{fiber}%
\DeclareMathOperator{\hocofiber}{hocof}%
\DeclareMathOperator{\hocof}{hocof}%
\DeclareMathOperator{\holim}{holim}%
\DeclareMathOperator{\hocolim}{hocolim}%
\DeclareMathOperator{\colim}{colim}%
\DeclareMathOperator{\Lan}{Lan}%
\DeclareMathOperator{\Ran}{Ran}%
\DeclareMathOperator{\Map}{Map}%
\DeclareMathOperator{\Id}{Id}%
\DeclareMathOperator{\mlf}{mlf}%
\DeclareMathOperator{\Hom}{Hom}%
\DeclareMathOperator{\Ho}{Ho}
\DeclareMathOperator{\Aut}{Aut}%
\DeclareMathOperator{\Stab}{Stab}
\DeclareMathOperator{\Iso}{Iso}
\DeclareMathOperator{\Ob}{Ob}

% ---------- shortcuts

\newcommand{\F}{\ensuremath{\mathcal F}}
\newcommand{\V}{\ensuremath{\mathcal V}}
\newcommand{\Q}{\ensuremath{\mathcal Q}}
\renewcommand{\O}{\ensuremath{\mathcal O}}
\renewcommand{\P}{\ensuremath{\mathcal P}}
\newcommand{\C}{\ensuremath{\mathcal C}}
\newcommand{\A}{\ensuremath{\mathcal A}}
\newcommand{\G}{\ensuremath{\mathcal G}}

\newcommand{\del}{\partial}%

\newcommand{\ki}{\chi}
\newcommand{\ksi}{\xi}
\newcommand{\Ksi}{\Xi}


% ---------- this paper

\newcommand{\Lie}{\mathrm{Lie}}
\newcommand{\Loday}{\mathcal L}
\newcommand{\redstar}{{\color{red}\**}}
\newcommand{\greenstar}{{\color{ForestGreen}\**}}



% %%%%%%%%%%%%%%%%%%%%%%%%%%%%%%%%%%%%%%%%%%%%%%%%%%%%%%%%%%%%%%%%%%%%%%%%%%%%%%%%%%%%%%%%%%%%%%%%%%%%
% ------------------------------ MAIN BODY ------------------------------

% ---- Title --------

\title{Proof of Theorem 5.4}

% \author{Eva Belmont, Peter Bonventre, Ang Li}%

\date{\today}


% ---- Document body ----

\begin{document}


% \begin{abstract}
%       Things and stuff
% \end{abstract}


 \maketitle

% \tableofcontents

\section{First definitions and comparisons}


Let $\Gamma$ denote the category of finite based sets and based maps, and $k$ a fixed ground field.
\begin{definition}
        A \textit{left $\Gamma$-module} is a functor $\Gamma \to k-mod$, while a
        \textit{right $\Gamma$-module} is a functor $\Gamma^{op} \to k-mod$.
\end{definition}

\begin{definition}
        Given a DGA $\Lambda$ and a $\Lambda$-module $R$, the \textit{(graded) Loday functor} is the left $\Gamma$-module
        \[
                \Loday(\Lambda, R) \colon \Gamma \to k-mod,
                \qquad
                [n] \mapsto \Lambda^{\otimes_R n}.
        \]
\end{definition}

\begin{definition}
        Let $\Lie_m^{\**}$ denote the \textcolor{red}{dual Lie representation}.
        We define a functor $\Ksi$ from left $\Gamma$-modules to bicomplexes by
        \[
                \Ksi(F)_{p,q} = \mathcal B \left( \Lie_{q+1}^{\**}, \Sigma_{q+1}, F(q+1) \right)
                = \Lie_{q+1}^{\**} \otimes k\left[\Sigma_{q+1}^{\times p}\right] \otimes F(q+1).
        \]
        One differential comes from the bar complex and is easy; the other is more complicated.
\end{definition}

\begin{definition}
        Given a DGA $\Lambda$ and a $\Lambda$-module $R$, the \textit{$\Gamma$-cohomology} of $\Lambda$ over $R$ is the bigraded abelian group
        \[
                H\Gamma^{\redstar, \greenstar}(\Lambda, R) := H^{\redstar} \mathrm{Tot} \Hom_R^{\greenstar} \left( \Ksi(\Loday(\Lambda, R)), R \right).
        \]
\end{definition}

\todo[inline]{The above clarifies the notation from both of \cite{RW02,Rob03} by resolving the ambigous placements of $\Ksi$ and also identifing the bidegrees.}



% \subsection{``Unnecessary'' definitions}

% Following [{\cite[\S 2.1]{Pir00}}],
% Any left $\Gamma$-module $F$ can be prolonged to a functor $F \colon \mathsf{Set}_{\**}^{\Delta^{op}} \to \mathsf{Vect}^{\Delta^{op}}$
% from the category of simplicial pointed sets to simplicial vector spaces,
% first by right Kan extending along the inclusion $\Gamma \to \Set_{\**}$ of finite pointed sets into all pointed sets,
% and then acting degreewise.
% Analogoulsy, by left Kan extending and acting degreewise, any right $\Gamma$-module $F$ can be prolonged to a functor
% $F \colon \mathsf{Set}_{\**}^{\Delta^{op}} \to \mathsf{Vect}^{\Delta}$ to cosimplicial vector spaces.

% \begin{claim}
%         For any left $\Gamma$-module $F$, there is a natural transformation $S^1 \wedge F(S^n) \to F(S^{n+1})$,
%         and we define the \textit{stable homotopy groups}
%         \[
%                 \pi_{\**}^{st}(F) := \colim \pi_{\** + n} F(S^n).
%         \]
%         Similarly, for any right $\Gamma$-module $F$ there is a natural transformation 
% \end{claim}




\begin{proposition}[{\cite[Cor 3.7]{Rob03} or \cite[Cor. 3.17]{Rob18}}]
        There is an isomorphism $\pi_{\redstar}(F) \cong H_{\redstar}\mathrm{Tot}\Ksi(F)$
        natural in left $\Gamma$-modules $F$.
\end{proposition}

\todo[inline]{We actually need the \textit{dual} of this, which I am having trouble formulating.
  Here is a guess, which really doesn't make sense, as $R$ is not defined for an arbitrary $F$}
\begin{claim}
        We have an isomorphism
        \[
                \pi^{\**}\Hom_R(F,R) \cong H^{\**}\mathrm{Tot} \Hom_R(\Ksi(F), R)
        \]
        natural in left $\Gamma$-modules $F$.
\end{claim}

\begin{lemma}[{\cite[\S 4.9]{Rob18}}]
        A homotopy ring spectrum $V = (V, \mu, \eta)$ determines a graded right $\Gamma$-module
        \[
                n \mapsto \pi_{\greenstar} \mathsf{End}(V)(n) = V^{\greenstar}(V^{\wedge n})
        \]
\end{lemma}

\begin{remark}
        Suppose $V$ satisfies a universal coefficient theorem:
        \begin{equation}
                \label{UCT_EQ}
                V^{\**}(V^{\wedge k}) \simeq \Hom_{V_{\**}}\left( (V_{\**}V)^{\otimes k}, V_{\**} \right).
        \end{equation}
        Then we have an isomorphism of graded right $\Gamma$-modules
        \[
                \pi_{\greenstar} \mathsf{End}(V) \equiv
                \Hom_{\V_{\**}}^{\greenstar}\left( \Loday(V_{\**}V, V_{\**}), V_{\**} \right). 
          \]
\end{remark}

\begin{corollary}
        \label{COMP_COR}
        For $V$ satisfing \cref{UCT_EQ},
        we have an isomorphism between $\Gamma$-homology of $V_{\**}V$ and the cohomotopy of $\pi_{\**} \mathsf{End}(V)$.
\end{corollary}
\begin{proof}
        We have
        \begin{align*}
          H\Gamma^{\redstar, \greenstar}(V_{\**}V, V_{\**})
          &= H^{\redstar} \mathrm{Tot} \Hom_{V_{\**}}^{\greenstar} \left( \Ksi \Loday(V_{\**}V, V_{\**}), V_{\**} \right) \\
          &= \pi^{\redstar} \Hom_R^{\greenstar} \left( \Loday(V_{\**}V, V_{\**}), V_{\**} \right) \\
          &= \pi^{\redstar} \pi_{\greenstar} \mathsf{End}(V) \\
          &= \pi^{\redstar} \left( V^{\greenstar}(V^{\wedge \bullet}) \right).
            \stepcounter{equation}\tag{\theequation}\label{COMP_EQ}\\
        \end{align*}
\end{proof}


\section{Results}

\subsection{Version 1}
\begin{theorem}[{\cite[Thm. 4.13]{Rob18}}]
        Let $V$ be a homotopy commutative ring spectrum.
        Then the obstruction to lifting an $n$-stage lives in $\pi^n \left( V^{2-n}(V^{\wedge \bullet}) \right)$. 
\end{theorem}

\begin{corollary}
        If $V$ is homotopy commutative and satisfies \cref{UCT_EQ}, then the obstruction lives in
        $H \Gamma^{n, 2-n}(V_{\**} V, V_{\**})$. \qed
\end{corollary}

\subsection{Version 2}
\begin{proposition}[{\cite[Prop. 5.4]{Rob03}}]
        For a homotopy commutative ring spectrum satisfying \cref{UCT_EQ},
        the obstruction lives in $H^n\mathrm{Tot} \Hom_{V_{\**}}^{2-n}(\Ksi \Loday(V_{\**}V, V_{\**}, V_{\**}))$.
\end{proposition}
\begin{proof}
        For each $1 \leq m \leq n+1$, we have an obstruction in
        \begin{align*}
                V^1(\nabla^{n+1}/(\nabla^n \cup \partial \nabla^{n+1}) \wedge_{\Sigma_m} V^{\wedge m})
          & =
            \Hom_{V_{\**}}^{2-n}(Lie_m^{\**} \otimes V_{\**}[\Sigma_m^{n-m+1}] \otimes \V_{\**}V^{\otimes m}, V_{\**}) \\
          & = 
            \Hom_{V_{\**}}^{2-n}\left( \Ksi_{n-m+1, m-1}\Loday(V_{\**}V, V_{\**}), V_{\**} \right).
        \end{align*}
        These combine to form an $n$-cochain in $\mathrm{Tot}\Hom_{V_{\**}V}^{2-n}(\Ksi \Loday(V_{\**}V, V_{\**}), V_{\**})$.
        \todo[inline]{so this is problem if the number of $\Sigma_m$-s is wrong, as we expect}
        {\color{red} the rest of the proof...}
\end{proof}

\begin{corollary}
        The obstruction lives in
        $H \Gamma^{n, 2-n}(V_{\**} V, V_{\**}) = \pi^n \left( V^{2-n}(V^{\wedge \bullet}) \right)$. \qed
\end{corollary}

\section{Questions}

\begin{question}
        Which side of \cref{COMP_EQ} is more conceptual? Easier to compute? Useful?
\end{question}



\bibliography{biblio}{}
\bibliographystyle{amsalpha2} 



\end{document}